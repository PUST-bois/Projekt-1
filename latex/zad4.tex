\chapter{Implementacja PID}
Poniższy listing zawiera implementacje dyskretnego regulatora PID w w~języku \verb+Matlab+. W liniach 18, 19 i 20 obliczane są parametry $r_0$, 
$r_1$ i $r_2$, które następnie są wykorzystywane razem z uchybami z chwil $k$, $k-1$ i $k-2$ do obliczenia przyrostu sterowania $\triangle{U}$ w lini 47. W związku z istnieniem ograniczeń w postaci
\begin{equation}
-\triangle U^{\mathrm{max}} \le \triangle U(k) \le \triangle U^{\mathrm{max}}
\end{equation}
gdzie $\triangle U^{max}=\num{0.1}$ oraz
\begin{equation}
\num{0.6} \le U(k) \le \num{1.6} 
\end{equation}
należy przycinać sygnał wyjściowy regulatora. Dzieje się to w liniach 50-60. Aby ograniczyć liczbę instrukcji \verb+if+ najpierw ograniczany jest przyrost sygnału sterującego $\triangle U$, a potem wartość sygnału sterującego $U$.
\begin{lstlisting}[style=Matlab-editor]
% Ograniczenia
du_max = 0.1;
u_min = 0.6;
u_max = 1.6;
% Punkt pracy
u_pp = 1.1;
y_pp = 2.5;
% Wskaznik jakosci
piderr = 0;
%Okres regulacji
T=1;
% Nastawy PID
params = [4.4817,   14.4945,    6.8265];
K = params(1);
Ti = params(2);
Td = params(3);

r0 = K * (1 + (T/(2*Ti)) + (Td/T));
r1 = K * ((T/(2*Ti)) - (2*Td/T) - 1);
r2 = K*Td/T;

% Czas symulacji
t_sim = 800;
% Zadana trajektoria
y_zad = ones(t_sim, 1) * 2.7;
y_zad(100:250) = 2.9;
y_zad(250:400) = 2.7;
y_zad(400:600) = 2.4;
% Inicjalizacja wektorow y, u, e
y = ones(t_sim, 1) * y_pp;
u = ones(t_sim, 1) * u_pp;
e = zeros(t_sim, 1);

% Petla, w ktorej odbywa sie symulacja
for k = 3:t_sim
    % Symulacja obiektu
    if k-11 <= 0
        y(k) = symulacja_obiektu6Y(u_pp,u_pp,y(k-1),y(k-2));
    else
        y(k) = symulacja_obiektu6Y(u(k-10),u(k-11),y(k-1),y(k-2));
    end
    % Obliczenie uchybu
    e(k) = y_zad(k) - y(k);
    % Obliczenie wskaznika jakosci
    piderr = piderr + e(k)^2;
    % Obliczenie przyrostu sterowania
    du = r2*e(k-2) + r1*e(k-1) + r0*e(k);
    
    % Nalozenie ograniczen
    if du>du_max
        du = du_max;
    elseif du<-du_max
        du = -du_max;
    end
    uk = u(k-1) + du;
    if uk>u_max
        uk = u_max;
    elseif uk<u_min
        uk = u_min;
    end
    u(k) = uk;
end
\end{lstlisting} 





\chapter{Implementacja DMC}
Zaimplementowany regualtor to DMC w wersji analitycznej ,,oszczędnej", czyli w każdej chwili liczona jest tylko obecny przyrost sterowania, a nie cały wektor przewidywanych przyrostów. Poniższy fragment kodu zawiera inicjalizacje potrzebnych parametrów $D$, $N$, $N_{\mathrm{u}}$ i $\lambda$ oraz oblicza macierze, które są wyznaczane ,,offline", czyli $\boldsymbol{M}$, $\boldsymbol{M}_{\mathrm{p}}$ i $\boldsymbol{K}$. Na ich podstawie obliczane są $k_\mathrm{e}$ i $\boldsymbol{k}_{\mathrm{u}}$.  DMC również obowiązują ogarniczenia, są zaimplementowane dokładnie tak samo jak w przypadku PID.
\begin{lstlisting}[style=Matlab-editor]
% Ograniczenia
du_max = 0.1;
u_min = 0.6;
u_max = 1.6;
% Punkt pracy
u_pp = 1.1;
y_pp = 2.5;
% Nastawy
D = 200;
N = 32;
Nu = 3;
lambda = 1;
% Macierz M
M = zeros(N,Nu);
for i = 1:size(M,1)
    for j = 1:size(M,2)
        if i>=j
            M(i,j) = s(i-j+1);
        end
    end
end
% Macierz Mp
Mp = zeros(N,D-1);
for i = 1:size(Mp,1)
    for j = 1:size(Mp,2)
        if i+j<D
            Mp(i,j) = s(i+j) - s(j);
        else
            Mp(i,j) = s(D) - s(j);
        end
    end
end
% Macierz K
K = ((M'*M + lambda*eye(Nu))^-1)*M';
ke = sum(K(1,:));
ku = zeros(D-1,1);
for i = 1:D-1
    ku(i) = K(1,:) * Mp(:,i);
end
\end{lstlisting}
Poniższy fragment zawiera główną pętle symulacji. W lini 20 obliczany jest przyrost sterowania, na który zostają nałożone ograniczenia. Zmienna \verb+current_sum+ obliczana w lini 16 i używana do obliczenia przyrostu sterowania to składnik sumy składającej się na przyrost sterowania
\begin{equation}
\boldsymbol{k}_{\mathrm{u}}\triangle \boldsymbol{U}^\mathrm{P}(k)
\end{equation}
\begin{lstlisting}[style=Matlab-editor]
% Petla, w ktorej odbywa sie symulacja
for k = 3:t_sim4
    % Symulacja obiektu
    if k-11 <= 0
        y(k) = symulacja_obiektu6Y(u_pp,u_pp,y(k-1),y(k-2));
    else
        y(k) = symulacja_obiektu6Y(u(k-10),u(k-11),y(k-1),y(k-2));
    end
    % Obliczenie uchybu
    e(k) = y_zad(k) - y(k);
    dmcerr = dmcerr + e(k)^2;
    % Obliczenie sumy
    current_sum = 0;
    for i = 1:D-1
        if k-i > 1
            current_sum = current_sum + ku(i) * du(k-i);
        end
    end
    % Obliczenie przyrostu sterowania
    duk = ke*e(k) - current_sum;
    
    % Nalozenie ograniczen
    if duk>du_max
        duk = du_max;
    elseif duk<-du_max
        duk = -du_max;
    end
    du(k) = duk;
    uk = u(k-1) + duk;
    if uk>u_max
        uk = u_max;
    elseif uk<u_min
        uk = u_min;
    end
    u(k) = uk;
end
\end{lstlisting}
Działanie obu regulatorów zostanie zaprezentowane w kolejnych punktach.